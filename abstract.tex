%% The following is a directive for TeXShop to indicate the main file
%%!TEX root = diss.tex

%Introduction
%RL->MDP, Q, SARSA, SMDP, 
%MAXQ (include the algorithm), hierarchical optimal vs recursive optimal, HORDQ, Relational,  Model-based RL (cite Peter Stone's Model based approach and a brief intro of it, include the algorithm)
%the problem of HO
%Optimal planning, model the differences
%scale to large problems
%The power of model-based RL and relational approach--> use a maze to illustrate this(model the difference)
%Linear SARSA
%Super Mario

%Motivation:
%1. scale to large problem with small number of samples

%chpater 2: 
%1. MaxQ

%chapter 3:
%1. The power of model based approach--> one wall sample and one pincecess can solve the whole maze problem
   %1.1. show the effectiveness
   %1.2. scaling up--> a noval biased model-based approach (show that it is the only way)
   %1.3. the necessity of biased approach
   %1.4. the choice between planning variable and not
   %1.5. wrong decision variable causes disater (decision)

%chapter 4:
%1. the three optimality ()
%2. Combined with hierarchical optimal RL
%3. why we need leaf cover
%4. the convergence for online approach
%5. the offiline one
%6. Respect the hierarchy

%chpater 5 scaling it up
%1. Better generalization: relational rl
%1.1. the power of model the difference
%2. Reduce training time: transfer learning
%3. case study: Super Mario

%chapter 6 future work
%1. make it hierarchical optimal (add only put down to get passenger subtask in taxi domain)
%2. A general way to combine any approximated model-based approach and can still achieve optimality
\chapter{Abstract}

Model-based reinforcement learning methods make efficient use of samples by
building a model of the environment and planning with it. Compared to
model-free methods, they usually take fewer samples to converge to the optimal
policy. Despite that efficiency, model-based methods may not learn the optimal
policy due to structural modeling assumptions. In this thesis, we show that by
combining model-based methods with hierarchically optimal recursive Q-learning (HORDQ)
under a hierarchical reinforcement learning framework, the proposed approach
learns the optimal policy even when the assumptions of the model are not all
satisfied. The effectiveness of our approach is demonstrated with Bus domain
and Infinite Mario -- a Java implementation of Nintendo's Super Mario Brothers.

%we propose a simple approach that assumes most of the variables are static
%during the planning process and focuses on modeling the few dynamic variables.

% Embed version information inline -- you should remove this from your
% dissertation
\vfill
\begin{center}
\begin{sf}
\fbox{Revision: ubcdiss.cls r27
}
\end{sf}
\end{center}
