\chapter{Conclusions}

%Our result suggests the possibility incorporating a wide range of planning
%techniques such as STRIPS [11] or object-oriented RL [16] into the HRL framework without the
%loss of optimality. One future direction is to investigate the applicability of combining our method
%with such techniques.

Model-based methods are powerful tools. They allow us to predict the outcome 
of the agent behaviour and plan over it. They can effectively reduce the number of 
samples which are required to find the optimal policy.
However, model-based methods may not be able to learn the optimal
policy due to the structural assumptions.
%However, model-based methods may not be able to learn the transition function correctly.
%The possible reasons are:
%\begin{itemize}
%\item The inaccuracy in the supervised learning algorithms
%\item The numerical imprecision in the continuous case
%\item It is not possible to learn all the effects
%\item It would be time-consuming if we compute all possible effects during planning process
%\end{itemize}
In this work, we propose an approach to combine the approximate model-based method with the
model-free method (HORDQ) under the HRL framework. We are able to show that our approach
can learn the optimal policy even when the assumptions of model are not satisfied. Furthermore, we show that optimality
is guaranteed for any subtask policy as long as the conditions of Theorem \ref{thm:opt} are satisfied.

%Bruno maybe it is a bit weird to talk about left cover in this point. I believe the reader might be
%a bit confused by this language. Perhaps you could in more abstract terms or remember the reader what a leaf means...
%James The reference of leaf cover is removed  

In this chapter, we share our experiences about how to apply our theory to design a system.
Since the performance of a system highly depends on the value of pseudo-rewards, we will also introduce
some heuristics for choosing an appropriate pseudo-reward.
Finally, we discuss limitations and possible directions for future works. 

%Bruno same comment above but here about pseudo-rewards. 
%James "pseudo-reward" is used across all chapters. The readers should have be familiar it. This is the last chapter after all.

%In this thesis, we describe an approach to use model-free methods to 
%compensate the scenarios when model-based methods are failed to learn the optimal policy.
%In small problems which we can use table-lookup methods, it is guaranteed to learn the optimal
%policy. 
%For large problems, there are no guarantee, but we should 
%The choice of features of model-free methods effect the overall performance
%If we put too much irrelevant features, the qualify of learned policy will be severed.

\section{System design}

%Bruno Re. the paragraph below, who said that design principle? Do you have a reference? Or are you claiming that this is true?
%James It is how I designed the features in the experiment, based on my experience. It is not a statement, so it cannot be true or false.

An important design principle of the system is to design the model-based method first,
and the model-free method later. 
We need to decide the features used by the
model-based methods, the underlying supervised learning algorithms and most
importantly, the effects which we would like to include in our model. Then we
run the experiments, and observe the scenarios where the model-based method fails. 
Based on the observations, we design a set of features for model-free methods to handle
these scenarios. Note that we don't need to design a set of features to handle
the whole problem, but only part of it. We only need the model-free method
to take control when the model-based method fails. Therefore, we can reduce the
number of features for model-free methods and let the overall system
successfully handle all scenarios.

%Bruno What do you mean by features above? If you include features as a new set of variables, the model-based method could fail in a different way, and you might loop in this specification...
%James The model-based methods and model-free methods use different features. The above process fixes the features of the model-based method, and changes the model-free features only.
%James The model-based method stays the same, so it will fail in the same way.

%//There are other alternative approaches to overcome their limitation.
Since our work is about how to use model-free methods to improve the learned policy of model-based methods,
it is not necessary to adopt our method if model-based methods can learn the optimal policy
on their own.

%Bruno Did you specify the conditions when model-based method will fail and your method will succeed?
%James                                                       ^^^^                               ^^^^
%James        It depends on the model-based method and the problem.
%James                                                   If it is possible to find a set of features which can represent a better policy, it will.

Our work is not the only solution when model-based methods fail.
Another alternative is to improve the quality of the model by including more domain knowledge.
For example, in our Infinite Mario experiment, we did not include any effects 
of the interactions between monsters and Mario. It is possible to
hand-code the preconditions and postconditions of these effects, as Walsh et al. proposed in \cite{Walsh09}.
In fact, the source code of Infinite Mario is publicly available.
%(TODO: source code)
There is no need to use model-based methods to learn the model. Instead,
we can simulate the experiences of the agent with the simulator of Infinite Mario. 
Since the environment and the model are identical, there are no biases which will be 
introduced during the simulation process.
This is what Baumgarten, the champion of Mario AI competition 2009, did with his $A^*$ method for Mario AI \cite{Robin09}.
With the complete domain knowledge, it is unlikely for any RL methods to outperform his work.

%Bruno No biases inotrudced by the simulation? But you are learning a policy right? That's a bias by definition...
%James The biases mean that there is a difference between simulation and the environment. Do you think I should change another verb? 

%Bruno unlikely for any RL to outperform his work? how do you know? That's a pretty strong statement. Did he claim that?
%James He is still the world champion. But the reason is obvious-- RL methods need to learn how to play the game, but his A* approach has the complete knowledge of Mario in the beginning,
%James so there are no learning phase in his approach. 

However, the key idea of RL is to build an adaptive agent.
Not only do we want the agent to perform well in a problem which we know very well,
but we also want the agent to adapt itself to novel problems which we cannot foresee when we design the agent.
If we put too much domain knowledge into the agent's design, we forbid it from adapting 
itself when the prior knowledge does not hold anymore. 
In this work, we introduce an alternative -- instead of designing an omnipotent model-based agent, we
divide the learning task into different parts and let model-free methods handle the parts which model-based methods cannot do.

%Bruno But you said that your methodology involved observing how the model-based fails and ONLY THEN designing the model-free.. That doesn't give you the authonomy you claim in this paragraph.
%James It is not possible to achieve the autonomy in RL. We need to specify the features, the primitive actions, the rewards and all the parameters. It is not possible to do these stuffs automatically.
%James The idea here is to develop a generic AI which knows as little domain knowledge as possible, so we can apply it to different problems. 
%James The engineering process such as designing a set of good features is still necessary, but it should be done in a generic way (like using image features only)
%James There is a trade-off between generality and effectiveness. Baumgarten's A* is an example of effectiveness, but the same AI code cannot be applied to Pacman, so the generality is limited.
%James The objective of my work is to achieve the best possible effectiveness and keep certain amount of generality at the same time.

%We want the agent to adapt itself to novel scenarios and find the optimal policy.
%If we put too much domain knowledge into the design of the model, the agent 
%will not be able to adapt when the knowledge does not hold in these scenarios.
%If we put too less, the model may be poor-approximated, 
%and hinder the learning process. 
%(TODO: my school bus exp)
%Our work tries to find a balance between 


%TODO: relationship with HRL (no 
%Although we use HRL framework to incorporate model-based and model-free methods, 
%TODO: the choice of model-free methods (not only HORDQ), anything that can know its consquences  
%It is not a problem to find a hierarchy

\section{Choosing an appropriate pseudo-reward}
It is important to choose an appropriate pseudo-reward. If we choose a pseudo-reward which is too small, 
the policy of the agent will be similar to the policy of a model-free method. Therefore, we may lose the 
benefit of the faster learning rate. On the other hand, if the pseudo-reward is too large, 
the policy will be similar to the model-model method, which may be suboptimal
when the assumptions of the model are not all satisfied.

It is easy to determine when a pseudo-reward is too large by
looking at the difference between the expected reward of the optimal policy and the policy of model-based method.
If a pseudo-reward is larger than this difference, the model-free method will follow the policy
of the model-based method strictly, and the combined policy will be the same as 
the policy of model-based method.

In our experiments (Sections \ref{se:BusRes} and \ref{se:MarioRes}), a pseudo-reward larger than the expected death penalty is considered "too large", 
since it will let the model-free method follow the instruction of model-based method even 
when it will result in the death of the agent. If we choose a pseudo-reward which is smaller than it,
the model-free method will choose an action that avoids the death of the agent.

It is more difficult to decide if the pseudo-reward is large enough. 
For small problems, if we adopt table-lookup HORDQ as the model-free method,
the optimal policy will be learned when the pseudo-reward is decreased to zero.
So a viable strategy is to choose some pseudo-reward, which is not too large, and
gradually decrease it to zero.
For large problems, we have adopted model-free methods with function approximation techniques
, therefore the optimal policy might not be learned when the pseudo-reward is zero.
Instead, we need to find out an optimal pseudo-reward which can maximize the expected reward.
A way to decide it is to conduct the experiment with the model-free method,
and choose a pseudo-reward which is large enough to encourage
the model-free method to follow the policy of model-based one.
%It is necessary to have a large pseudo-reward in the beginning since 
%the model-based method does not have enough samples to build a good model,
%it is possible for it to compute very poor policy.

\section{Limitations and future work}

The quality of learned policy depends on the chosen model-free method,
the model-based method and the pseudo-reward. 
Since we can control the pseudo-reward to decide if the combined
policy should be similar to model-free or model-based one,
the combined policy can never be worse than any of them. 

Since our work is a combination of the two, it will fail in scenarios where both methods fail. 
We could only improve on this scenario if we apply better model-free or model-based methods. 

%If we adopt the task hierarchy in \ref{se:MarioExp}, we can apply our work
%to any reinforcement learning problems.  
%The limitation of the applicability
%of our work actually comes from the underlying RL methods. 

Nevertheless, our work is not useful when one method outperforms another.
In general, model-based methods learn faster than model-free methods because of
the efficient use of samples, but it may not be true for some problems. 
If the chosen model-based method is worse than the model-free method in both learning
rate and the learned policy, it is pointless to combine both methods. 
Similarly, if the model-free method fails to handle the scenarios where the model-based
method fails or the model-based method can learn the optimal policy, it is not necessary to apply our work.
Our work has its edge when the model-based method learns faster than the model-free method but learns
a worse policy compared to the model-free one.
For small problems, if the model-based method fails to learn the optimal policy, 
we learn the optimal policy by combining it with table-lookup HORDQ as we prove in Theorem \ref{thm:opt}.
For large problems, it is difficult since approximated model-free RL may not learn the optimal policy.
It is necessary to have the knowledge about the domain and apply the knowledge to choose some good features.

%Bruno The 1st line of the previous paragraph is confusing. Clearly one method will outperform the other.
%James There are two objectives of RL algorithms: 1. Learn the best policy 2. in minimal steps
%James It is possible that one method outperforms another in one objective but fails in another 

%TODO: the process is similar to model-free RL
We introduced the theory of improving the quality of the policy of model-based methods. However, we don't have any theory regarding the learning rate.
It is true that if the model-based method is "approximately good", we can enjoy the 
faster learning rate, as we showed in the Bus domain experiment. Nevertheless, there is no theory to tell if a model-based method is approximately good or not.
A possible direction of future work is to investigate what kind of properties of model-based
methods are necessary to increase the learning rate and how much they can increase when they are combined with model-free methods.

%Bruno Your use of the word "learning rate" might mean different things to different people. Did you define it somewhere in your thesis? I'm not reading in order so I don't know it at this point.
%James No, I didn't define it. Do you know any paper which provides the definition?

%Bruno Perhaps you could get a better closure for the thesis. I get the feeling that you are ending it too suddenly. But this is a minor style thing...
%James You are right. Any ideas what I should say here?

%1. No theory about learning rate (what is a good approximation?)
%2. No thery about the interaction betwwen model-based and model-free approach.
%If model-base dapproahc does not change it's policy accroding to model-free's policy,
%the learning can be slower than pure model-free. Since model-based method
%will stick on it's original bad plan, and guid the agent to the wrong action.

%A problem may arise when the pseudo-reward is 

%The performance of our approach depends on 
%Large enough so that the model-free approach will pursue the subgoal
%When the model-based approach doesn’t have enough samples to build an approximately good model
%Small enough to pass the control to model-free approach when the model-based approach doesn’t work

%Combine both model-free and model-based approaches
%Overcome the structure assumption of model-based approaches
%Learns a better policy by mixing these approaches
%Designing pseudo-reward requires insight of the problem
%No theory yet!

%model-based or model-free is too good

%We have presented a learning algorithm for agile, integrated whole-body
%skills of physically-simulated characters. The algorithm uses a nature-inspired
%online, active exploration of the character action space to nd reliable mo-
%tions that give rise to parameterized skills. We further demonstrated that
%our algorithm works for a family of simple characters without requiring any
%algorithm or parameter modications. In addition, we experimented with
%a complex dog character in 3-D and showed that our approach generalizes
%to this character, given appropriate changes in the motor abstractions used
%during the learning process. Finally, we showed that the resulting parame-
%a terrain.
%5.1 Discussion
%While a learning approach to acquiring skills possesses many benets, it also
%comes with its own set of limitations.
%Mainly, we found that the learning process requires occasional super-
%vision to ensure that the intended skills are actually being learned. For
%example in one case, a character learned a Flip motion that made it launch
%necessary to supervise the learning process and restart it on a few occasions.
%The majority of skill and character combinations (roughly 90%) did not re-
%quire any interventions. We believe that these issues could be alleviated by
%better quantifying when a particular trial should count as being successful

%for every skill.
%In addition, the phase 1 reward functions may be dicult to specify in
%some cases. In particular, we found the GetUp skill to be the most trou-
%blesome. If the character fails at getting up after some Motor Action, how
%should one assign a score for how close the attempt was? Specifying a bad
%phase 1 reward function could lead to long computation times in phase 1,
%because the character is essentially left searching randomly in the motor
%space for a successful action. Even worse, the optimization in phase 1 could
%be repeatedly led astray with an inappropriately specied phase 1 reward
%function.
%The main challenges for the quadruped character were aesthetic in na-
%ture. Unlike the Acrobot's motions, a dog's leap is a specic type of motion
%that we are all familiar with from nature. Even though the learning al-
%gorithm produced leaps that accomplished all desired goals, they did not
%always resemble leaps that one would expect to see from a real dog. For
%but this motion did not emerge in the actions that were produced by the
%algorithm. Instead, the dog left its front feet outstretched during the leap,
%producing a motion that felt qualitatively strange despite achieving all task
%goals. In the end, we opted to include these details into the leap controller
%manually to achieve a more familiar style of motion.
%5.2 Limitations and future work
%Even though the learning algorithm described in this thesis works well for
%our characters and the set of skills we considered, we make no claims to
%have addressed the general problem of motor learning. In this section, we
%discuss possible extensions of the proposed framework that can bring us
%closer toward the nal goal of matching human or animal abilities.
%To begin with, several immediate improvements can be made to the
%framework by addressing some of the simplications that were made mostly

