%% The following is a directive for TeXShop to indicate the main file
%%!TEX root = diss.tex

\chapter{Introduction}
\label{ch:intro}

%hard to learn a model: non-smothness of probability and reward function, factoered assumtion, size of planning envelope
%cannot give any optimality guarantees,
Reinforcement learning (RL) addresses the problem of finding an optimal policy in a stochastic environment.
In the RL setting, an agent interacts with the environment, optimizing its behavior to
maximize the received rewards.
RL methods can be broadly classified into two classes: model-based and
model-free.  Model-based RL learns an effective policy by
constructing the model from samples and simulating experiences from the model. It
generally requires fewer samples to learn the optimal policy. However, when the
state space is too large, we cannot build the exact model anymore. Instead, we
have to approximate the model with techniques such as function approximation. Little 
research has been done to address the problem of model-based reinforcement learning with approximation in
an online setting.
%TODO

Methods based on factored Markov decision processes (FMDPs) assume the problem has some factored structure, and use 
specialized algorithms that exploit the structure \cite{ApproxFactor} \cite{SPUDD}. 
However, these methods require prior knowledge
of the structure of the problem, which may not be available in practice.
Degris and Sigaud \cite{ApproxTree} extended Dyna \cite{Dyna} with the approximation of decision trees.
Sutton et al. \cite{ApproxDyna} introduced linear Dyna -- a combination 
of Dyna and linear function approximation. 
Instead of enumerating all states, which is impossible for large problems, they
use Dyna-style planning. Given the current state and policy, 
their methods predict the features of the next state and reward, using function approximation
techniques. Despite the difficulty of correct prediction, their methods are biased 
in a stochastic environment. It is possible to have
several possible next states given the same state and policy in a stochastic environment.
If we predict the most likely one, we will not find the optimal policy because of the bias in our model.

On the other hand, model-free methods learn the Q-function directly. 
There are no simulation steps for model-free methods and modeling is unnecessary. 
The existing linear function approximation algorithms for model-free methods have been successfully 
applied to large domains \cite{LSTD99}\cite{KeepAway}. 

In this work, we combine model-based and model-free methods with the hierarchical 
reinforcement learning (HRL) framework. We propose a simple approximate approach which 
only enumerates a small number of states during planning, thus it is possible to apply it to large 
domains. 
However, the model is too simple to represent complex problems and it may not be able to represent the 
optimal policy. We show that by combining it with 
hierarchically optimal recursive Q-learning (HORDQ) \cite{HORDQ}, which is a model-free method, we can 
guarantee that the overall policy will converge to the optimal one even when our model
fails to approximate the problem. Our approach assumes the task hierarchy for an MDP is given. The hierarchy can either be 
designed manually or learned by automatic hierarchy discovery techniques \cite{HexQ}.

%Our contribution includes: 
%1. Derived a condition that the hierarchical optimal policy is equal to the optimal policy
%2. Under the same condition, some subtasks policies do not affect the optimality of overall policy
%3. A fail safe mechanism to ensure that approximated model-based approach may

%TODO: state that it is fine if we do not learn the transition function or reward function correctly

We are not the first to try to combine different RL methods within the HRL framework.  
Ghavamzadeh and Mahadevan \cite{HybridPolicy} combined value function-based RL and policy gradient RL to handle
continuous MDP problems. Cora \cite{Vlad} incorporated model-based, model-free and Bayesian active learning into the MAXQ framework.
Nevertheless, these methods seek recursive optimality in the learning process, 
thus they fail to satisfy any optimality condition when one of the subtasks
fails to find its optimal policy.
In contrast, our method learns the optimal policy without the requirement that 
all of the policies of the subtasks need be optimal. It is thus more robust and allows us to incorporate approximate
approaches into the same framework.


%say the benefit of model-based RL (peter's model based HRL, 
%illustrate the problem of model-based RL

%show that it can be resolved by HRL
%say other HRL work, but no one addresses the optimality with biased model

%At higher level, we use model-based approach to do the planning on a coarser state representation. 
%At lower level, model-free approach is used....
%[Ditterich did that as well]
%Instead of focusing on safe state abstraction, we are more interested in the unsafe one.
%The state of real world problem can be very large and complicated, we may 
%not always find a safe abstracition. In this work, we show that the 
%optimality can still be achieved if 1. 2. 
%[CMU TR][] some uses (unsafe) corasers state at higher level, but none of them 
%provide any guarrante the optimality when the coarser representation are
%not safe, 

%None of the previous approaches 

